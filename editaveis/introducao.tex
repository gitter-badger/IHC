\chapter[Introdução]{Introdução}
	
	Durante o processo de construção de um produto, em muitos cenários, não são considerados a pespectiva do usuário em relação a utilização desse produto. Do ponto de vista da engenharia, o produto pode funciona normalmente, mas para o usuário final, quem tem a visão mais realista da utilização do produto, pode não atender as espectativas propostas.
	Dentro deste contexto temos o conceito de design de interação, no qual consiste em relacionar os aspectos de usabilidade com o processo de design do produto, ou seja, enfatizar experiência do usuário, facilidade de uso, eficiencia, entre outros, todos atendendo a necessidade do usuário. Faz parte do processo entender aspectos relacionados à interação em tarefas que dão suporte às atividades cotidianas.
	Este trabalho contem a construção de um processo de design de interação para um aplicativo de notificação de preços de passagens aéreas (MyPush Travel). Nele há a definição e a aplicação de ferramentas, técnicas e conceitos para o levantamento de um protótipo e consolidação de uma idéia centrada no usuário.