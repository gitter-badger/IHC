\chapter[Medidas de Satisfação]{Medidas de Satisfação}
\section{Tecnicas de Interação Humano-Computador}
No design de interface de usuário deve ser considerado fatores que facilite a utilização da aplicação, além de ser atraente para o usuário. Para isso é necessário aplicar técnicas das quais é possível medir a eficiência, eficacia e satisfação do usuário. Isso é chamado de usabilidades.

As medidas de usabilidade podem ser obtidas por:

\begin{itemize}
	\item Medidas objetivas: está relacionada ao desempenho do usuário enquanto usa a interface. Promovem dados de tempo, velocidade ou ocorrencia de eventos particulares.
	\item Medidas subjetivas: representam a opnião do usuário para com a interface. São dados de sentimentos, atitudes, preferência.
\end{itemize}

\section{Técnica de Questionamento}

A técnica de questionamento identifica se os requisitos estão de acordo com a necessidade do usuaŕio. Uma técnica simples e barata de administrar e demonstram resultados significativos que podem mostrar necessidades que o designer não havia considerado.
Há dois modos de aplicar um questionamento: entrevista e questionários.

\section{Questionário}

 A vantagem do questionário em cima da entrevista, é que pode ser aplicado em um grande número de pessoas, além de apresentar dados quantitativos. É possível utilizar o questionário em várias fases do processo de design.
 Os questionários devem ser bem elaborados, compreensíveis e que gerem resultados que são adequados para a análise.

 \section{Coleta dos Dados}

 Avaliar subjetivamente com um questionário é umportante para a avaliação de usabilidade. Por isso foram elaborados modelos de questionários para essa finalidade. Os mais conhecidos são:

 \begin{itemize}
	\item QUIS (Questionnaire for User Interaction Satisfaction);
	\item SUMI (Software Usability Measurement Inventory);
	\item WAMMI (Website Analysis and MeasurMent Inventory)
	\item SUS (System Usability Scale).
\end{itemize}

\subsection{QUIS}
Desenvolvido por uma equipe multidisciplinar de pesquisadores do HCIL (Human-Computer Interaction Laboratory) da University of Maryland. Pode ser configurada de acordo com a necessidade de análise de cada interface. Dividido em seções onde cada uma especifica algum ponto de interesse da interface.
As questões são feitas em forma de afirmações e utilizam uma escala diferencial semântica, de 0 a 9, onde zero apresenta o total negativo e 9 o total positivo.