\chapter[Metas e princípios de usabilidade]{Metas e princípios de usabilidade}

	Considerando o contexto desse trabalho foram considerados as seguintes metas 
de usabilidade e experiência do usuário.

\textbf{Metas de Usabilidade:}
\begin{itemize}
	\item Ser fácil de aprender
	\item Ser fácil de lembrar como se usa
\end{itemize}	

\textbf{Experiência do Usuário:}
\begin{itemize}
	\item Ser agradável
	\item Ser útil
\end{itemize}


	Ainda tendo em mente facilitar a interação do usuário com o MyPush as seguintes
heurísticas de usabilidade também serão atendidas.

\textbf{Controle de usuário de liberdade}
\begin{itemize}
	\item O sistema deve permitir configuração das datas e horários da maneira que melhor lhe atender, por exemplo, em quais dias será realizada a pesquisa, escolher se será mensal ou semanal ou ainda configurar o formato da hora 12h ou 24h.
\end{itemize}

\textbf{Consistência e padrões}
\begin{itemize}
	\item Deverá apresentar padrões relacionados a aplicativos mobile, por exemplo, menu de opções no canto superior direito.
\end{itemize}

\textbf{Estética e design}
\begin{itemize}
	\item O sistema deverá ser simples e minimalista, contendo a navegação das funcionalidades pelo menu principal.
\end{itemize}

\textbf{Prevenção de erros}
\begin{itemize}
	\item O sistema não irá permitir que o usuário informe uma data que já passou, informar uma mensagem de erro caso não haja espaço suficiente na memória para realizar o backup.
\end{itemize}