\chapter[Metas do Design]{Metas do Design}

Um ponto a se deixar claro quanto ao projeto de design é deixar claro qual é o objetivo principal. Portanto é necessário definir as metas que podem ser divididas em duas categorias:
\begin{itemize}
	\item \textbf{Metas de usabilidade}: tem o objetivo de atender aos critérios específicos da usabilidade (ex.: eficiência, seguro, útil). Geralmente considera assegurar que o produto é facil de usar e eficiente da pespectiva do usuário.
	\item \textbf{Metas decorrentes da experiência do usuário}: tem o objetivo de expor a qualidade da experiência do usuário(ex.: esteticamente agradável)
\end{itemize}

\section{Metas de Usabilidade e Experiência do Usuário}

A usabilidade são divididas nas seguintes metas (NIELSEN, 1993):
\begin{itemize}
	\item \textbf{Eficácia}: ser eficaz no uso. O quanto um sistema é bom em realizar o que se espera dele.
	\item \textbf{Eficiência}: ser eficiente no uso. Maneira como o sistema auxilia os usuários na realização da sua tarefa.
	\item \textbf{Segurança}: ser segura no uso. Proteger o usuário de situações perigosas e indesejáveis.
	\item \textbf{Utilidade}: ser de boa utilidade. Medida na qual o sistema propicia o tipo certo de funcionalidade para que o usuário realize o que deseja.
	\item \textbf{\textit{Learnability}}: ser fácil de aprender. O quão fácil é aprender a usar o sistema.
	\item \textbf{\textit{Memorability}}: ser fácil de lembrar como se usa. Facilidade de lembrar como utilizar o sistema.
\end{itemize}

Considerando o contexto desse trabalho foram considerados as seguintes metas de usabilidade e experiência do usuário.

\begin{table}[H]
	\centering
	\begin{tabular}{p{6cm}|p{6cm}}
		\toprule
			Meta de Usabilidade & Meta de experiência do usuário\\ \hline
		\midrule
			Ser fácil de aprender: O usuário deve sentir o mínimo de necessidade em pedir ajuda ou suporte na utilização do sistema & Ser agradável: Interface não deve conter poluição visual, deve ser simples e intuitiva. \\ \hline
			Ser fácil de lembrar como se usa. O usuário deve conseguir executar as atividades quantas vezes forem necessárias sem dificuldades na repetição. & Ser Satisfatório: Ser atendido pelo propósito da aplicação e ao ser auxiliado nas atividades envoltas do aplicativo. 
	\end{tabular}
	\caption{Tabela de metas}
	\label{tab01}
\end{table}

\section{Princípio de Usabilidade}
Os princípios de usabilidade, ou também conhecidos como heurísticas, são utilizados como base para avaliação de protótipos e sistemas existentes (NIELSEN, 2001).
Abaixo algumas heurísticas estabelecidas para o projeto.

\textbf{Controle de usuário de liberdade}
\begin{itemize}
	\item O sistema deve permitir configuração das datas e horários da maneira que melhor lhe atender, por exemplo, em quais dias será realizada a pesquisa, escolher se será mensal ou semanal ou ainda configurar o formato da hora 12h ou 24h.
\end{itemize}

\textbf{Consistência e padrões}
\begin{itemize}
	\item Deverá apresentar padrões relacionados a aplicativos mobile, por exemplo, menu de opções no canto superior direito.
\end{itemize}

\textbf{Estética e design}
\begin{itemize}
	\item O sistema deverá ser simples e minimalista, contendo a navegação das funcionalidades pelo menu principal.
\end{itemize}

\textbf{Prevenção de erros}
\begin{itemize}
	\item O sistema não irá permitir que o usuário informe uma data que já passou, informar uma mensagem de erro caso não haja espaço suficiente na memória para realizar o backup.
\end{itemize}