\chapter[Planejamento de Avaliação]{Planejamento de Avaliação}

\section{Sobre Avaliação}

	Será avaliado o usuário e suas tarefas, assim como observar, medir e analisar o seu desempenho com o sistema. 
	“Assim como os designers não deveriam assumir que todos são como eles, também não deveriam presumir que seguir as recomendações (guidelines) para o design seja garantia de uma boa usabilidade.”(PREECE, 2005, p.339, grifos do autor).
	Um avaliação bem feita serve para além de conhecimento. Alguns esclarecimentos são apontados pelo bem-sucedido designer Bruce Tognazzini, segundo Preece(2005, p.341) \nocite{jeniffer05}:

\begin{enumerate}
	\item Problemas são concertados antes de o produto ser lançado, não após.
	\item A equipe pode se concentrar em problemas reais, não em imaginários.
	\item Os engenheiros codificam, em vez de debater.
	\item O tempo para que o produto entre no mercado é menor.
\end{enumerate}

\section{Tipos de avaliações usadas}

Existem diferentes tipos de avaliações, cada qual coleta um determinado tipo de dado que pode ser utilizado para representar parte do comportamento do sistema do ponto de vista do usuário. As avaliações devem responder inicialmente questões chaves na hora se serem executadas: "para quem", "onde?" e "quais os resultados?". Com esses quesitos foram selecionados algumas formas de avaliação.

\subsection{Rápido-Sujo}
O paradigma usado para a avaliação será o “Rápido	 e sujo”. Na qual o usuário é observado com o seu comportamento natural.
Os avaliadores não terão o mínimo de controle sobre os usuários. A vantagem desta abordagem é que pode ser feita qualquer momento, obtendo assim um feedback rápido, como por exemplo, o design, técnicas de outros paradigmas podem ser usadas sem problemas.
Os dados colhidos com o paradigma “Rápido e sujo”, são do tipo qualitativos ou descrições informais. Esses dados são retornados ao design em forma de esboços, citações e relatórios descritivos.

\subsection{Observação Direta}

Consiste na observação dos usuários com o intuito de identificação de necessidades segundo os aspectos decorrentes da experiência do usuário na utilização do protótipo. Podem ser usados anotações, áudio, vídeos e logs de interação para registro das observações. Além disso também são utilizadas algumas técnicas para aumentar a quantidade de dados para análise, como \textit{Think-aloud} que consiste em pensar em voz alta, ou seja, enquanto o usuário realiza uma atividade no sistema ele deve falar em voz alta o que está pensando durante o processo. Solicitar opiniões dos usuários: “Consiste em pergunta ao usuário o que eles pensam a respeito do produto – se ele realiza o que querem; se eles o apreciam; se é esteticamente atraente;”(PREECE, 2005, p.366).

\subsection{Questionário}

Os questionários são ferramentas para coletar dados da avaliação. Tem o objetivo de capturar experiência e as opniões do usuário com relação ao protótipo e podem ser feitas com questões fechadas, abertas, escalaveis, etc. Existem diferentes modelos de questionários, cada qual com o objetivo de coletar elementos de uma área específica, seja com o intuito de identificar possiveis melhoras contra concorrentes de mercado ou apenas avaliações de interface e análises de fácil utilização.

\subsection{Entrevista}

Geralmente a entrevista é utilizada em conjunto com a observação direta, pois após, ou durante, a interação do usuário com o sistema é possível realizar perguntas pertinentes ao sistema que são utilizadas como insumo na melhoria do protótipo. Podem ser realizadas com questionários com questões fechadas ou abertas.

\section{Ciclo Avaliativo}

O ciclo básico de avaliação consite em quatro atividade básicas:

\begin{itemize}
	\item Preparação: preparar o ambiente, o protótipo, a escolha da quantidade e perfil dos usuários que realizarão a atividade de avaliação.
	\item Execução: utilizar das técnicas de avaliação com o protótipo e coletar os dados.
	\item Interpretação: organizar e analizar os resultados obtidos da avaliação.
	\item Relatar: elaborar um relatório com todos os dados e análises
\end{itemize}

Antes do início do primeiro ciclo avaliativo, é necessário realizar a elaboração inicial dos protótipos de baixa fidelidade. Com a primeira versão finalizada a técnica de "Rápido-Sujo" é utilizada para melhorar e levantar mais requesitos do sistema.

\section{Planejamento da avaliação do projeto My Push}

A avaliação ocorrerá em três ciclos avaliativos, a primeira iteração ocorrá com um protótipo de papel, previamente válido por meio de uma avaliação rapido-sujo, tal avalidação servirá para consolidar os requisitos e realizar melhorias no prótotipo de papel. A segunda iteração dar-se-á com uma ferramenta iterativa, no qual, fará uso do protótipo de papel e ira simular a aplicação por meio da lincagem das páginas do prótotipo. A ultima iteração irá utilizar um protótipo de alta fidelidade.

A seleção do público alvo baseou-se no perfil de um usuário que realiza diversas viagens e que costuma monitorar os preços das passagens com antecencia com o objetivo de obter os melhores preços. Para cada iteração será intrevistado 5 pessoas. Segundo (NIELSEN, 1993) cinco participantes podem revelar até 80 \% dos erros encontrados no projeto, alguns estudos como \textit{Why Five Users Aren't Enough} (Woolrych \& Cockton, 2001) apontam que 5 usuários não são suficientes, pois presumem por exemplo, que um determinado estudo pode apenas avaliar usuários de uma determinada faixa etária, e que provavelmente não conseguiriam achar a maior quantidade de erros existentes. Considerando que a avaliação desse projeto será focado em um perfil específico, pessoas que realizam pesquisa de preço de passagens aéreas, sem delimitar por exemplo, gênero ou faixa etária, acredita-se que cinco usuários para cada iteração será um número suficiente.

As técnicas de avaliações abordadas foram entrevistas com observação direta mais e o questionário. A escolha do questionário bem como será realizada a entrevista será explicada na seção 9 - Medidas de satisfação. Os resultados serão consolidados em um relatório de avaliação das metas de usabilidade, ele apresentara as violações das metas encontradas bem como a solução para as mesmas, um modelo do relatório se encontra no apêndice.

