\chapter[Planejamento de Avaliação]{Planejamento de Avaliação}

\section{Sobre Avaliação}

	Será avaliado o usuário e suas tarefas, assim como observar, medir e analisar o seu desempenho com o sistema. 
	“Assim como os designers não deveriam assumir que todos são como eles, também não deveriam presumir que seguir as recomendações (guidelines) para o design seja garantia de uma boa usabilidade.”(PREECE, 2005, p.339, grifos do autor).
	Um avaliação bem feita serve para além de conhecimento. Alguns esclarecimentos são apontados pelo bem-sucedido designer Bruce Tognazzini, segundo Preece(2005, p.341):

\begin{enumerate}
	\item Problemas são concertados antes de o produto ser lançado, não após.
	\item A equipe pode se concetrnar me problemas reais, não em imaginários.
	\item Os engenheiros codificam, em vez de debater.
	\item O tempo para que o produto entre no mercado é menor.
\end{enumerate}

\section{Tipos de avaliações usadas}

	O paradigma usado para a avaliação será o “Rápida e sujo”. Na qual o usuário é observado com o seu comportamento natural. Os avaliadores não terão o mínimo de controle sobre os usuários. A vantagem desta abordagem é que pode ser feit a qualquer momento, obtendo assim um feedback rápido, como por exemplo, o design, técnicas de outros paradigmas podem ser usadas sem problemas. 
	Os dados colhidos com o paradigma “Rápida e suja”, são do tipo qualitativos ou descrições informais. Esses dados são retornados ao design em forma de esboços, citações e relatórios descritivos.

\section{Técnicas} 

	Técnicas usadas para licitação de informação tanto do usuário quanto dos especialsitas no momento de avaliação:
Observação dos usuários: com o intuito de identiifcação de necessidades e avaliação de protótipos. Podem ser usados anotações, áudio, vídeos e logs de interação para registro das observações. 
Solicitar opiniões dos usuários: “Consiste em pergunta ao usuário o que eles pensam a respeito do produto – se ele realiza o que querem; se eles o apreciam; se é esteticamente atraente;”(PREECE, 2005, p.366). 
%	Solicitar opiniões dos especialistas: “Guiados por heurísticas, os especialistas desempenhavam as tarefas como se fossem usuários típicos, identificando problemas.”(PREECE, 2005, p.366).
