\chapter[Projeto de Design de Interação]{Projeto de Design de Interação}

Os objetivos de um projeto de design de interação é tornar simples a utilização do seu produto a quem o utiliza. Define a comunicação humana com o artefato ou produto interativo, que inclui decisões sobre o detalhes dos procedimentos de comunicação lógica ou física e comportamental do produto.
O projeto de interação tem aspectos específicos que dão diretrizes ao desenvolvimento do produto. As atividades devem estar centrada no usuário. O projeto deve compreender quais as atividades dão suporte ao usuário, quais as necessidades dele, identificar os requisitos do produto e verificar se é realmente o que o usuário deseja e precisa. Trata-se de uma construção de um conhecimento lógico e em comum entre os desenvolvedores e os usuários finais.

\section{Atividade do Projeto de Interação}

Em aspectos gerais das atividade do projeto de interação, pode-se observar quatro atividade básicas que definem o processo do projeto de interação:

\begin{itemize}
	\item \textbf{Identificar necessidades e estabelecer requisitos}: são a base do projeto e destacam os requisitos do produto. Essa atividade é de maior importância, pois ela que sustentam o design e desenvolvimento. É fundamental que seja centrada no usuário, pois é nessa etapa que se conhece o usuário alvo e a aplicação do produto para o seu contexto.
	\item \textbf{}:
\end{itemize}	