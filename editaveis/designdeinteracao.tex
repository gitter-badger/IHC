\chapter[Projeto de Interação]{Projeto de Interação}

Os objetivos de um projeto de interação é tornar simples a utilização do seu produto a quem o utiliza. Define a comunicação humana com o artefato ou produto interativo, que inclui decisões sobre o detalhes dos procedimentos de comunicação lógica ou física e comportamental do produto.
O projeto de interação tem aspectos específicos que dão diretrizes ao desenvolvimento do produto. As atividades devem estar centrada no usuário. O projeto deve compreender quais as atividades dão suporte ao usuário, quais as necessidades dele, identificar os requisitos do produto e verificar se é realmente o que o usuário deseja e precisa. Trata-se de uma construção de um conhecimento lógico e em comum entre os desenvolvedores e os usuários finais.
Deve-se considerar algumas características chaves no projeto de interação:
\begin{itemize}
	\item \textbf{Foco no usuário}: projetar e avaliar com foco na visão do usuário, ou seja, simular ou mesmo realizar tarefas como o usuário.
	\item \textbf{Critérios específicos de usabilidade}: objetivos devem está claramente documentado, pois ajudam em ter visão das diferentes alternativas do projeto.
	\item \textbf{Iterãção}: repetição, ou seja, o processo de testes e avaliações devem ser repetidos para que o projeto tenha refinamentos durante o desenvolvimento, com base no \textit{feedback} do usuário.
\end{itemize}

\section{Atividade do Projeto de Interação}

Em aspectos gerais das atividade do projeto de interação, pode-se observar quatro atividade básicas que definem o processo do projeto de interação:

\begin{itemize}
	\item \textbf{Estabelecer necessidades e requisitos}: são a base do projeto e destacam os requisitos do produto. Essa atividade é de maior importância, pois ela que sustentam o design e desenvolvimento. É fundamental que seja centrada no usuário, pois é nessa etapa que se conhece o usuário alvo e a aplicação do produto para o seu contexto.
	\item \textbf{Desenvolver projetos alternativos}: nessa etapa é onde surgem as idéias que devem atender aos requisitos. O modelo conceitual do produto ganha forma juntamente com a descrição sobre o que o produto fará, como se comportará e parecerá.
	\item \textbf{Construir versões interativas}: como o usuário verificará se as necessidade e requisitos estão sendo atendidas? É nessa etápa que são construidos versões dos protótipos para mostrar ao usuário como o produto está sendo modelado.
	\item \textbf{Avaliar o que está sendo construido}: onde são executadas formas de determinar o grau de usabilidade e aceitabilidade do produto. É verificado se parte do produto encontra-se em condição de uso.
\end{itemize}

Nas próximas seções serão especificados as atividades do projeto de interação dentro do contexto do projeto.